\documentclass[journal, 11pt,twocolumn]{IEEEtran}
\usepackage[utf8]{inputenc}
\usepackage{hyperref}
\usepackage{amsmath,amssymb}
\usepackage{graphicx}
\begin{document}
\title{\huge ICSE 10 2017 PAPER}
\author{\Small Pundi Bindusree\\ \small cs21btech11048  }
\maketitle
\vspace{-12pt}

\section*{\Small Question 5@ solution:}

Given a matrix $ B = \begin{bmatrix}
1 & 1 \\
8 & 3 \\
\end{bmatrix} $
and a matrix \(X\) such that $X = B^2 - 4B $
\begin{align}
   B^2 & = \begin{bmatrix}
   1 & 1\\
   8 & 3\\
\end{bmatrix}
\nonumber\\
          & = \begin{bmatrix}
   1 \times 1 + 1 \times 8 & 1 \times 1 + 1 \times 3\\
   8 \times 1 + 3 \times 8 & 8 \times 1 + 3 \times 3\\
 \end{bmatrix}
 \nonumber\\
          & = \begin{bmatrix}
    1 + 8 & 1 + 3\\
    8 + 24 & 8 + 9\\
   \end{bmatrix}
   \nonumber\\
          & = \begin{bmatrix}
    9 & 4\\
    32 & 17\\
    \end{bmatrix}\\
4B & = \begin{bmatrix}
   4 \times 1 & 4 \times 1\\
   4 \times 8 & 4 \times 3\\
        \end{bmatrix}
        \nonumber\\
        & = \begin{bmatrix}
    4 & 4\\
    32 & 12\\
    \end{bmatrix}
    \end{align}\\  substituting $(1)$ and $(2)$ in $X = B^2 - 4B$ gives\begin{align}
    X & = \begin{bmatrix}
    9 & 4\\
    32 & 17\\
    \end{bmatrix}-\begin{bmatrix}
    4 & 4\\
    32 & 12\\
    \end{bmatrix}\nonumber\\
    & = \begin{bmatrix}
    9 - 4 & 4 - 4\\
    32 - 32 & 17 - 12\\
    \end{bmatrix}\nonumber\\
    & = \begin{bmatrix}
    5 & 0\\
    0 & 5\\
    \end{bmatrix}\nonumber
    \end{align}\\ Therefore, $X = \begin{bmatrix}
    5 & 0\\
    0 & 5\\
    \end{bmatrix}$
    
 Given that $X\begin{bmatrix}
    a\\
    b\\
    \end{bmatrix}=\begin{bmatrix}
    5\\
    50\\
    \end{bmatrix}$
    \begin{align}
  X\begin{bmatrix}
  a\\
  b\\
  \end{bmatrix} & = \begin{bmatrix}
  5 & 0\\
  0 & 5\\
  \end{bmatrix}
 \begin{bmatrix}
 a\\
 b\\
 \end{bmatrix}
 \nonumber\\
                            & = \begin{bmatrix}
                            5 \times a + 0 \times b\\
                            0 \times a + 5 \times b\\
                            \end{bmatrix}
                            \nonumber\\
                            & = \begin{bmatrix}
                            5a+0\\
                            0+5b\\
                            \end{bmatrix}
                            \nonumber\\
                            & = \begin{bmatrix}
                            5a\\
                            5b\\
                            \end{bmatrix} = \begin{bmatrix}
                            5\\
                            50\\
                            \end{bmatrix}
                            \end{align}as \begin{align}
                            \begin{bmatrix}
        5a\\
        5b\\
        \end{bmatrix} & = \begin{bmatrix}
        5\\
        50\\
        \end{bmatrix}
        \nonumber
        \end{align}\\
        
       comparing L.H.S values in left side matrix with R.H.S values in right side matrix in (3) gives
\begin{align}
5a & = 5\nonumber\\
5b & = 50\nonumber
\end{align}\\
from $5a=5,$ $a = 1$ and $5b=50,$ $b=10.$\\
Therefore from above the values of $a$,$b$ are $1$,$10$ respectively.
   \end{document}
